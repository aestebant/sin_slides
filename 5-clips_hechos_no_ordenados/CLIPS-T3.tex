\documentclass[usenames,dvipsnames,aspectratio=169]{beamer}
\usepackage[utf8]{inputenc}
\usepackage[spanish]{babel}
\usepackage{csquotes}
\usepackage[skip=0pt]{caption}
\captionsetup[figure]{labelformat=empty}
\usepackage{booktabs}
\usepackage{tabularx, colortbl}
\usepackage{setspace}
\usepackage{hyperref}
\usetheme{uniud}

%%% Some useful commands
% pdf-friendly newline in links
\newcommand{\pdfnewline}{\texorpdfstring{\newline}{ }} 
% Fill the vertical space in a slide (to put text at the bottom)
\newcommand{\framefill}{\vskip0pt plus 1filll}
\renewcommand{\emph}[1]{\textbf{\textcolor{UniGold}{#1}}}


\title[]{{\LARGE Sistemas Inteligentes\\CLIPS}\\[0.5cm]Tema 3: Hechos II}

\date[Marzo, 2022]{14 de marzo de 2022}
\author[A. Esteban]{\texorpdfstring{
    \begin{minipage}{0.47\linewidth}
        Aurora Esteban Toscano
        \pdfnewline
        \texttt{aestebant@uco.es}
    \end{minipage}
}{Aurora Esteban Toscano}
}
\institute{Grado en Ingeniería Informática, Universidad de Córdoba}
\begin{document}

%\setstretch{1.5}

\begin{frame}
\titlepage
\end{frame}

\begin{frame}{Objetivos}
	\begin{itemize}
		\item Profundizar en los \textit{hechos no ordenados} en CLIPS.
		\item Profundizar en la afirmación de hechos en CLIPS.
		\item Conocer los comandos de gestión de hechos en CLIPS.
	\end{itemize}
\end{frame}

\begin{frame}[fragile]{Introducción a los hechos en CLIPS}
	Los hechos representan la información actual del Sistema Experto: determinan cómo actúa y se actualizan según las condiciones cambian (las distintas reglas se disparan).

	Tipos de hechos:
	\begin{itemize}
		\item Ordenados: son una relación de valores. Por ejemplo:\\
		\texttt{(lista huevos leche ``pimiento rojo'')}
		\item No ordenados: su estructura se determina por una plantilla.
		%\texttt{(coche (marca Seat) (modelo Ibiza) (color amarillo))}
	\end{itemize}
	\begin{minipage}{.35\textwidth}
		\begin{minipage}{.9\textwidth}
			Los hechos no ordenados permiten representar información con la relación entre conceptos y atributos.
		\end{minipage}
	\end{minipage}
	\begin{minipage}{.55\textwidth}
		\begin{exampleblock}{Por ejemplo}
			\footnotesize
			\begin{verbatim}
(deftemplate coche ``Es un coche''
    (slot marca)
    (slot modelo)
    (slot color))
(assert (coche (marca Seat)))
(assert (coche (marca Audi) (modelo A1) (color rojo)))
			\end{verbatim}
		\end{exampleblock}
	\end{minipage}
\end{frame}

\begin{frame}[fragile]{Propiedades de las plantillas de hechos}
	\begin{block}{Constructor}
		\begin{verbatim}
(deftemplate <nombre> [comentario]
      (slot|multislot <nombre> [atributos*])+
)
		\end{verbatim}
	\end{block}
	\begin{itemize}
		\item Las plantillas deben tener nombre único: si se define una plantilla con el mismo nombre que otra ya existente, la sobrescribirá.
		\item Las plantillas no se pueden redefinir mientras estén en uso: si un hecho o una regla se base en ella.
		\item Las plantillas se componen de un numero variable de \textit{campos}:
		\begin{itemize}
			\item Monocampos o campos simples: \texttt{slot}.
			\item Multicampos o campos compuestos: \texttt{multislot}.
		\end{itemize}
	\end{itemize}
\end{frame}

\begin{frame}[fragile]{Ejemplo: \texttt{persona.clp}}
	\begin{enumerate}
		\item Descargar de Moodle el fichero y observar la estructura de la plantilla.
		\begin{verbatim}
(deftemplate persona "datos de una persona"
     (slot nombre)
     (slot edad)
     (multislot direccion)
)
		\end{verbatim}
		\item Cargar la plantilla en CLIPS.
		\item Afirmar los hechos:
		\begin{itemize}
			\item Persona cualquiera de 34 años.
			\item Juan, de 26 años.
			\item Marta, de 21 años y dirección Avd de Cervantes.
		\end{itemize}
	\end{enumerate}
\end{frame}

\begin{frame}{Configuración de los campos en plantillas I}
	\begin{itemize}
		\item Valor por defecto. \texttt{(slot <nombre> (default|default-dynamic <por defecto>))}
		\begin{itemize}
			\item \texttt{default}: valor por defecto estático:
			\begin{itemize}
				\item \texttt{?DERIVE}: derivado de las restricciones del campo. Comportamiento por defecto.
				\item Puede ser un valor primitivo.
				\item \texttt{?NONE}: forzar que no se derive $\Rightarrow$ campo obligatorio.
			\end{itemize} 
			\item \texttt{default-dynamic}: valor por defecto dinámico. La expresión es evaluada cada vez que se crea una instancia.
		\end{itemize}
		\item Tipo del campo. \texttt{(slot <nombre> (type <tipo>))}
			\begin{itemize}
				\item Se permiten los tipos de datos primitivos: \texttt{INTEGER, FLOAT, NUMBER, SYMBOL, STRING, LEXEME}
			\end{itemize}
	\end{itemize}
\end{frame}

\begin{frame}{Configuración de los campos en plantillas II}
	\begin{itemize}
		\item Valores permitidos. \texttt{(slot <nombre> (allowed-... ))}
		\begin{minipage}{.4\textwidth}
			\begin{itemize}
				\item \texttt{allowed-values}
				\item \texttt{allowed-numbers}
				\item \texttt{allowed-lexemes}
			\end{itemize}
		\end{minipage}
		\begin{minipage}{.4\textwidth}
			\begin{itemize}\setlength\itemsep{1pt}
				\item \texttt{allowed-integers}
				\item \texttt{allowed-floats}
				\item \texttt{allowed-symbols}
				\item \texttt{allowed-strings}
			\end{itemize}
		\end{minipage}
		\item Rango de valores permitido. \texttt{(slot <nombre> (range <min> <max>))}
		\item Cardinalidad permitida en campos múltiples. \texttt{(multislot <nombre> (cardinality <min> <max>))}
	\end{itemize}
\end{frame}

\begin{frame}[fragile]{Ejemplo: \texttt{equipo.clp}}
	\begin{enumerate}
		\item Descargar de Moodle el fichero y observar la estructura de la plantilla.
\begin{verbatim}
(deftemplate equipo "datos de un equipo de futbol"
    (slot id (default-dynamic (gensym*)))
    (slot nombre (type STRING) (default ?NONE))
    (slot posicion (type INTEGER) (range 1 10) (default 5))
    (multislot equipacion (type SYMBOL) (allowed-symbols blanco rojo
    azul verde amarillo))
    (multislot plantilla (type LEXEME) (cardinality 3 7))
)
\end{verbatim}
		\item Cargar la plantilla en CLIPS. Cargar los hechos definidos en \texttt{liga}
		\item ¿Qué pasa con el campo \texttt{id}? ¿Y con el campo \texttt{nombre}?
		\item ¿Cuál es la posición máxima que puede ocupar un equipo?
		\item ¿Cuántos jugadores pueden componer un equipo?
	\end{enumerate}
\end{frame}

\begin{frame}{Otras gestiones sobre plantillas}
	\begin{itemize}
		\item \texttt{(list-deftemplates)}: listar las plantillas usadas por el sistema.
		\item \texttt{(ppdeftemplate <nombre>)}: muestra la definición de una plantilla existente.
		\item \texttt{(undeftemplate <nombre>)}: borra una plantilla existente.
	\end{itemize}
\end{frame}

\begin{frame}{Afirmación de hechos}
	Dos modalidades:
	\begin{itemize}
		\item \texttt{assert}: introducir en la base de hechos los hechos dinámicamente. Por ejemplo:\\
		\texttt{(assert (a b c))}\\
		\texttt{(assert (a) (b) (c))}\\
		\texttt{(assert (a b) (c d) (e))}
		\item \texttt{deffacts}: inicializar la base de hechos al principio. Por ejemplo:\\
		\texttt{(deffacts h1 ``hechos iniciales'' (a b) (c d) (e))}
		\begin{itemize}
			\item La base de hechos no se inicializará hasta ejecutar \texttt{(reset)}
		\end{itemize}
		\item OjO: no se puede insertar un hecho que ya existe.
	\end{itemize}
\end{frame}

\begin{frame}[fragile]{La orden \texttt{deffacts}}
	\begin{block}{Constructor}
		\begin{verbatim}
(deffacts <nombre> [comentario]
    <hecho>+
)
		\end{verbatim}
	\end{block}
	\begin{itemize}
		\item Pueden existir múltiples constructores \texttt{deffacts}, cada uno con cualquier número de hechos tanto ordenados como no ordenados.
		\item Si se carga un nuevo \texttt{deffacts} con el mismo nombre de otro existente, sobrescribe a los hechos que éste definiera.
		\item Puede contener hechos con expresiones dinámica. El hecho tomará el valor de la expresión resultante.
	\end{itemize}
\end{frame}

\begin{frame}{Otras gestiones sobre \texttt{deffacts}}
	\begin{itemize}
		\item \texttt{(list-deffacts)}: listar los constructores de hechos definidos.
		\item \texttt{(ppdeffacts <nombre>)}: muestra la definición de un constructor de hechos dado.
		\item \texttt{(undeffacts <nombre>)}: suprime los hechos afirmados en un constructor de hechos dado.
	\end{itemize}
\end{frame}

\begin{frame}[fragile]{Por ejemplo}
	\footnotesize
	\begin{verbatim}
CLIPS> (clear)
CLIPS> (deffacts EstadoCoche (coche motor "a punto") (coche bujias "limpias"))
CLIPS> (assert (coche neumaticos "gastados"))
<Fact-1>
CLIPS> (reset)
CLIPS> (facts)
f-1     (coche motor "a punto")
f-2     (coche bujias "limpias")
CLIPS> (assert (coche luces "funcionan"))
<Fact-3>
CLIPS> (deffacts EstadoCasa (casa salon "ordenado"))
CLIPS> (reset)
CLIPS> (facts)
f-1     (coche motor "a punto")
f-2     (coche bujias "limpias")
f-3     (casa salon "ordenado")
CLIPS> (list-deffacts)
EstadoCoche
EstadoCasa
	\end{verbatim}
\end{frame}

\begin{frame}
\titlepage
\end{frame}
\end{document}