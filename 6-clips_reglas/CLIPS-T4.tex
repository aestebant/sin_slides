\documentclass[usenames,dvipsnames,aspectratio=169]{beamer}
\usepackage[utf8]{inputenc}
\usepackage[spanish]{babel}
\usepackage{csquotes}
\usepackage[skip=0pt]{caption}
\captionsetup[figure]{labelformat=empty}
\usepackage{booktabs}
\usepackage{tabularx, colortbl}
\usepackage{setspace}
\usepackage{hyperref}
\usetheme{uniud}

%%% Some useful commands
% pdf-friendly newline in links
\newcommand{\pdfnewline}{\texorpdfstring{\newline}{ }} 
% Fill the vertical space in a slide (to put text at the bottom)
\newcommand{\framefill}{\vskip0pt plus 1filll}
\renewcommand{\emph}[1]{\textbf{\textcolor{UniGold}{#1}}}


\title[]{{\LARGE Sistemas Inteligentes\\CLIPS}\\[0.5cm]Tema 4: Reglas I}

\date[Marzo, 2022]{21 de marzo de 2022}
\author[A. Esteban]{\texorpdfstring{
    \begin{minipage}{0.47\linewidth}
        Aurora Esteban Toscano
        \pdfnewline
        \texttt{aestebant@uco.es}
    \end{minipage}
}{Aurora Esteban Toscano}
}
\institute{Grado en Ingeniería Informática, Universidad de Córdoba}
\begin{document}

%\setstretch{1.5}

\begin{frame}
\titlepage
\end{frame}

\begin{frame}{Objetivos}
	\begin{itemize}
		\item Introducción a las reglas en CLIPS.
		\item Conocer el ciclo básico de ejecución de reglas.
		\item Profundizar en la definición y las propiedades de las reglas.
	\end{itemize}
\end{frame}

\begin{frame}{Introducción a las reglas en CLIPS}
	Si el \textbf{antecedente} es cierto según los hechos almacenados en la \textbf{base de afirmaciones}, entonces \textbf{pueden} realizarse las acciones especificadas en el \textbf{consecuente}.
	\begin{block}{Estructura de las reglas}
		\centering
		\textbf{Antecedente} $\Longrightarrow$ \textbf{Consecuente}
	\end{block}
	\begin{itemize}
		\item Antecedente: cero o más \textit{cláusulas} que deben cumplirse para que la regla pueda ejecutarse (dispararse).
		\item Consecuente: cero o más \textit{acciones} que se llevarán a cabo si la regla se dispara.
		\begin{itemize}
			\item Entre esas acciones puede estar crear (afirmar) más hechos, eliminar hechos obsoletos, llegar a conclusiones finales...
		\end{itemize}
	\end{itemize}
	\begin{exampleblock}{Por ejemplo}
		\centering
		\texttt{SI} voy a clase \texttt{Y} atiendo \texttt{Y} hago los ejercicios $\Longrightarrow$ apruebo la asignatura
	\end{exampleblock}
\end{frame}

\begin{frame}{Gestión de reglas}
	Elementos de los Sistemas Expertos relacionados con la gestión de reglas:
	\begin{itemize}
		\item Base de conocimiento: conjunto de reglas que describen el problema a resolver.
		\item Motor de inferencia: comprueba el antecedente de las reglas y la base de afirmaciones, decide la próxima regla a ejecutar y aplica su consecuente.
		\item Activación o disparo de reglas: las reglas se disparan una única vez para un mismo conjunto de antecedentes $\Rightarrow$ evitar entrar en bucles.
		\begin{itemize}
			\item Entidad patrón: elementos en los que se basa la activación de una regla. Pueden ser hechos ordenados, plantillas e instancias de clases.
			\item Elemento condicional (EC): son cada una de las condiciones que pueden encadenarse para componer el antecedente de la regla. 
		\end{itemize}
	\end{itemize}
\end{frame}

\begin{frame}[fragile]{Definición de las reglas I}
	\begin{block}{Constructor}
		\begin{verbatim}
(defrule <nombre> [comentario] [propiedades]
      <elemento-condicional>*
      =>
      <acción>*
)
		\end{verbatim}
	\end{block}
	\begin{itemize}
		\item Las reglas deben tener nombre único: si se define una regla con el mismo nombre que otra ya existente, la sobrescribirá.
		\item No hay límite en el número de EC ni de acciones que una regla tenga.
		\begin{itemize}
			\item Si no hay ningún EC en el antecedente $\Rightarrow$ la regla se ejecuta automáticamente.
			\item Si no hay ninguna acción en el consecuente $\Rightarrow$ ejecución sin consecuencias.
		\end{itemize}
	\end{itemize}
\end{frame}

\begin{frame}{Definición de las reglas II}
	\begin{itemize}
		\item Las reglas se \textbf{activan} cuando se satisface su antecedente.
		\begin{itemize}
			\item Las reglas pueden activarse para distintos conjuntos de hechos $\rightarrow$ cada activación es una \textit{instancia de regla}.
		\end{itemize}
		\item Las instancias de reglas se almacenan en la \textbf{agenda}: muestra las reglas disponibles para ejecutar.
		\begin{itemize}
			\item Para cada instancia de regla muestra: la prioridad de la regla, el nombre y los índices de los hechos que la satisfacen.
		\end{itemize}
		\item Las reglas se \textbf{disparan} cuando el motor de inferencia decide ejecutar las acciones de sus consecuentes.
		\item Se puede fijar la \textbf{prioridad} de una regla entre [-10000, 10000]. Prioridad por defecto: 0.
		\item La \textbf{estrategia de resolución de conflictos} decide qué regla disparar la siguiente si hay varias con la misma prioridad. 
	\end{itemize}
\end{frame}

\begin{frame}[fragile]{Ejemplo}
	\footnotesize
		\begin{verbatim}
CLIPS> (clear)
CLIPS> (defrule hola1 => (printout t "Hola Mundo" crlf))
CLIPS> (defrule hola2 (initial-fact) => (printout t "Hola Mundo iniciado" crlf)))
CLIPS> (rules)
hola1
hola2
CLIPS> (agenda)
0      hola1: *
CLIPS> (assert (initial-fact))
CLIPS> (agenda)
0      hola2: f-1
0      hola1: *
CLIPS> (run)
Hola Mundo iniciado
Hola Mundo
CLIPS> (facts)
f-1     (initial-fact)
CLIPS> (agenda)
CLIPS>
		\end{verbatim}
\end{frame}

\begin{frame}{Ciclo básico de ejecución de reglas}
	\begin{enumerate}
		\item Las reglas activadas se disparan con el comando \texttt{(run [pasos])}
		\item Si se ha alcanzado el máximo de pasos $\Rightarrow$ fin de la ejecución.
		\item Se actualiza la agenda según la lista de hechos.
		\item Se selecciona la próxima instancia de regla a ejecutar según prioridades y estrategia de resolución de conflictos.
		\item Se dispara la instancia seleccionada, se incrementa el número de pasos y se elimina de la agenda.
		\item Se vuelve al paso 2.
	\end{enumerate}
\end{frame}

\begin{frame}{Ejemplo: \texttt{animales.clp}}
	El ejercicio muestra una serie de reglas que, en base a un hecho, indican qué sonido hace un animal.
	\begin{enumerate}
		\item Descarga de Moodle el fichero y observa la estructura.
		\item Carga el fichero en CLIPS:
		\begin{enumerate}
			\item Establece el directorio de trabajo a la ruta donde esté el fichero guardado. \texttt{Environment > Set Directory...}
			\item Carga el fichero ejecutando \texttt{(load animales.clp)}
			\item Inicializa la base de afirmaciones.
		\end{enumerate}
		\item ¿Cómo se ampliaría la base de conocimiento para contemplar otros animales como ovejas, vacas o patos?
		\item ¿Cómo se adaptarían las reglas para trabajar con hechos no ordenados?
	\end{enumerate}
\end{frame}

\begin{frame}{Ejemplo: \texttt{coche.clp}}
	El ejercicio muestra cómo se puede actualizar la información de la base de hechos por medio de reglas, tanto sobre hechos ordenados como no ordenados.
	\begin{enumerate}
		\item Descarga de Moodle el fichero y observa la base de conocimiento (dos reglas) y la base de afirmaciones (dos hechos).
		\item ¿Cuál será el flujo de inferencia?
		\item Carga el fichero en CLIPS e inicializa la base de afirmaciones.
		\item Ejecutar el motor de inferencia en pasos de uno en uno y ver el resultado.
	\end{enumerate}
\end{frame}

\begin{frame}[fragile]{Propiedades de las reglas}
	\begin{block}{Definición}
\begin{verbatim}
(defrule <nombre> [comentario]
     (declare <propiedad>)
     <antecedente> => <consecuente>
)
\end{verbatim}
	\end{block}
	\begin{itemize}
		\item Las propiedades permiten alterar el comportamiento en la activación de las reglas.
		\item La declaración de propiedades se incluye en la definición de la regla, después del comentario y antes del antecedente mediante la sentencia \texttt{(declare ...)}.
		\item Una regla puede tener una sola sentencia \texttt{(declare)}.
	\end{itemize}
\end{frame}

\begin{frame}{Propiedades de las reglas: prioridad}
	\begin{center}
		\texttt{(declare (salience <prioridad>))}
	\end{center}
	\begin{itemize}
		\item La prioridad determina en qué orden se disparan varias reglas activadas.
		\item Puede variar en el rango [-10000, 10000] $\rightarrow$ a mayor número, mayor prioridad y antes se dispara la regla.
		\item Valor de prioridad por defecto: 0.
		\item Evaluación de la prioridad
		\begin{itemize}
			\item Prioridad estática: en la definición de la regla. Comportamiento por defecto.
			\item Prioridad dinámica: (1) cuando se activa la regla o (2) en cada ciclo de ejecución.
		\end{itemize}
	\end{itemize}
\end{frame}

\begin{frame}[fragile]{Ejemplo}
	\begin{verbatim}
CLIPS> (clear)
CLIPS> (defrule regla1 (declare (salience 10))
    (initial-fact) => (printout t "Me ejecuto la primera" crlf)
)
CLIPS> (defrule regla2 (declare (salience -10))
    (initial-fact) => (printout t "Me ejecuto la segunda" crlf)
)
CLIPS> (assert (initial-fact))
CLIPS> (agenda)
10     regla1: f-1
-10    regla2: f-1
CLIPS> (run)
Me ejecuto la primera
Me ejecuto la segunda
	\end{verbatim}
\end{frame}

\begin{frame}{Ejercicio \texttt{pila.clp}}
	\begin{enumerate}
		\item Descarga de Moodle el fichero \texttt{pila.clp}, que implementa una estructura de datos tipo \textit{pila}.
		\item Observa la estructura de las operaciones \textit{push} y \textit{pop} y su comportamiento una vez se ejecuta el programa.
		\item ¿Por qué una regla tiene más prioridad que otra?
		\item ¿Cómo se implementaría una estructura de datos tipo cola?
		\item Si en la misma base de conocimiento coexisten las dos estructuras, ¿cómo indicamos al SE cuál de ellas queremos utilizar?
	\end{enumerate}
\end{frame}

\begin{frame}{Otras gestiones sobre reglas}
	\begin{itemize}
		\item \texttt{(list-defrules)}: listar las reglas definidas.
		\item \texttt{(ppdefrule <nombre>)}: muestra la definición de una regla dada.
		\item \texttt{(undefrule <nombre>)}: suprime la regla dada.
	\end{itemize}
	Además, de recordatorio...
	\begin{itemize}
		\item \texttt{(load <fichero.clp>)}: carga un script en CLIPS.
		\item \texttt{(facts)}: muestra la base de hechos o afirmaciones.
		\item \texttt{(rules)}: muestra la base de conocimiento.
		\item \texttt{(agenda)}: muestra la agenda.
		\item \texttt{(reset)}: borra la base de hechos, reinicia los hechos iniciales y reinicia la agenda.
		\item \texttt{(clear)}: borra toda la información del sistema.
		\item \texttt{(exit)}: salir de CLIPS.
	\end{itemize}
\end{frame}

\begin{frame}
\titlepage
\end{frame}
\end{document}