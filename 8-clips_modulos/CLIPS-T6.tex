\documentclass[usenames,dvipsnames,aspectratio=169]{beamer}
\usepackage[utf8]{inputenc}
\usepackage[spanish]{babel}
\usepackage{csquotes}
\usepackage[skip=0pt]{caption}
\captionsetup[figure]{labelformat=empty}
\usepackage{booktabs}
\usepackage{tabularx, colortbl}
\usepackage{setspace}
\usepackage{hyperref}
\usetheme{uniud}

%%% Some useful commands
% pdf-friendly newline in links
\newcommand{\pdfnewline}{\texorpdfstring{\newline}{ }} 
% Fill the vertical space in a slide (to put text at the bottom)
\newcommand{\framefill}{\vskip0pt plus 1filll}
\renewcommand{\emph}[1]{\textbf{\textcolor{UniGold}{#1}}}


\title[]{{\Large Sistemas Inteligentes\\CLIPS}\\[0.2cm]Tema 6: Módulos}

\date[]{Abril de 2023}
\author[Aurora Esteban]{\texorpdfstring{
    \begin{minipage}{0.47\linewidth}
        Aurora Esteban Toscano
        \pdfnewline
        \texttt{aestebant@uco.es}
    \end{minipage}
    \hfill
    \begin{minipage}{0.47\linewidth}
        José Manuel Alcalde Llergo
        \pdfnewline
        \texttt{i72alllj@uco.es}
    \end{minipage}
}{Aurora Esteban Toscano}
}
\institute{Grado en Ingeniería Informática, Universidad de Córdoba}

\begin{document}

\begin{frame}
\titlepage
\end{frame}

\begin{frame}{Objetivos}
	\begin{itemize}
		\item Conocer la estructuración en \textit{módulos} de CLIPS.
		\item Extender el conocimiento del ciclo de ejecución a la presencia de varios módulos.
		\item Dirigir el ciclo de ejecución entre módulos.
	\end{itemize}
\end{frame}

\begin{frame}[fragile]{Introducción a los módulos en CLIPS}
	Los módulos en CLIPS permiten facilitar el control del razonamiento de forma modular agrupando constructores (hechos, reglas, etc.)
	
	\begin{block}{Definición}
		\begin{verbatim}
(defmodule <nombre> [comentario]
      <(export <port-item)>*
      <(import <modulo> <port-item>)>*
)
<port-item>:: ?ALL | ?NONE | <constructor> ?ALL | <constructor> ?NONE |
           <constructor> <nombre>
<constructor>:: deftemplate | defclass | defglobal | deffunction | deffgeneric
		\end{verbatim}
	\end{block}
\end{frame}

\begin{frame}{Propiedades de los módulos}
	\begin{itemize}
		\item Siempre existe un módulo por defecto \texttt{MAIN}.
		\item El \textit{foco} determina cuál es el \textit{módulo activo}, es decir, el modo en ejecución.
		\item Los módulos sólo pueden borrarse con la orden \texttt{(clear)}
		\item Para especificar a qué módulo pertenece un elemento de CLIPS, dos aproximaciones:
		\begin{itemize}
			\item Referencia explícita: \texttt{<módulo>::<elemento>}.
			\item Referencia implícita: establecer el \textit{foco} en el módulo previamente, con lo que todos los elementos para los que no se especifique lo contrario se asumirán que están en el \textit{módulo activo}.
		\end{itemize}
		\item \texttt{(get-current-module)}: muestra el módulo actual.
		\item \texttt{(set-current-module <módulo>)}: cambia al módulo especificado.
	\end{itemize}
\end{frame}

\begin{frame}[fragile]{Ejemplo}
	\small
	\begin{minipage}[t]{0.47\textwidth}
		\begin{verbatim}
CLIPS> (clear)
CLIPS> (defmodule A)
CLIPS> (assert (dato 55))
<Fact-1>
CLIPS> (defmodule B)
CLIPS> (assert (dato 33))
<Fact-2>
CLIPS> (defrule busca55
(dato 55) => )
CLIPS> (facts)
f-2     (dato 33)
For a total of 1 fact.
		\end{verbatim}
	\end{minipage}
	\hfill
	\begin{minipage}[t]{0.47\textwidth}
		\begin{verbatim}
CLIPS> (agenda)
CLIPS> (list-defmodules)
MAIN
A
B
For a total of 3 defmodules.
CLIPS> (set-current-module A)
B
CLIPS> (facts)
f-1     (dato 55)
For a total of 1 fact.
CLIPS> (rules)
CLIPS> 
		\end{verbatim}
	\end{minipage}
\end{frame}

\begin{frame}{Importación y exportación de elementos}
	Los elementos exportables entre módulos se tienen que definir:
	\begin{enumerate}\small
		\item Como exportables en el módulo de origen.
		\item Como importables en el módulo de destino.
	\end{enumerate}
	Posibles ejemplos:
	\begin{itemize}\footnotesize
		\item \texttt{(defmodule A (export ?ALL))} / \texttt{(defmodule B (export ?NONE) (import A deftemplate x))}
		\item \texttt{(defmodule A (export deftemplate ?ALL))} / \texttt{(defmodule B (import A deftemplate ?ALL))}
		\item \texttt{(defmodule A (export ?ALL))} / \texttt{(defmodule B (export ?ALL))} / \texttt{(defmodule C (import A deftemplate datoA) (import B deftemplate datoB))}
	\end{itemize}
	\begin{block}{OjO}
		Los hechos de un módulo pueden exportarse a otro mediante su \texttt{deftemplate}. Las operaciones que se hagan sobre un hecho tienen efecto en todos los módulos en los que esté.
	\end{block}
\end{frame}

\begin{frame}{Ciclo de ejecución con módulos I}
	\begin{itemize}
		\item Cada módulo tiene su propia \textbf{base de conocimientos} y su propia \textbf{agenda} no compartible con otros.
		\item \texttt{(reset)}: restablece las bases de hechos de todos los módulos y devuelve el foco al módulo por defecto \texttt{MAIN}.
		\item \texttt{(clear)}: restablece completamente el sistema y devuelve el foco al módulo por defecto \texttt{MAIN}.
		\item \texttt{(run)} ejecuta la agenda del módulo activo $\rightarrow$ las instancias de sus propias reglas que se hayan activado para su propia base de hechos.
		\item Los módulos se organizan en una pila: conforme un módulo acaba pasa al siguiente de la pila $\rightarrow$ el programa termina cuando no quedan módulos activos.
		\begin{itemize}
			\item \texttt{(get-focus-stack)}: muestra el estado actual de la pila de módulos activos.
		\end{itemize}
	\end{itemize}
\end{frame}

\begin{frame}{Ciclo de ejecución con módulos II}
	Formas de alterar el ciclo de ejecución normal:
	\begin{itemize}
		\item Orden \texttt{focus} en el consecuente de una regla: incluye un nuevo módulo en la pila de módulos activos y pasa el control a la agenda de ese módulo.
		\item Orden \texttt{return} en consecuente de una regla: termina antes de tiempo con la agenda del módulo actual para pasar el control a la del siguiente módulo activo en la pila.
	\end{itemize}
\end{frame}

\begin{frame}{Ejemplo}
	\begin{itemize}
		\item Descarga de Moodle el fichero de fichero \texttt{modulos1.clp}. Observa la estructura y cárgalo en CLIPS. Ve ejecutando la agenda de uno en uno, atendiendo a cómo cambian las bases de afirmaciones y el foco del programa.
		\item Descarga de Moodle el fichero de fichero \texttt{modulos2.clp}. Observa la estructura y cárgalo en CLIPS. Ve ejecutando la agenda de uno en uno, atendiendo a cómo cambian las bases de afirmaciones y el foco del programa.
	\end{itemize}
\end{frame}

\begin{frame}{Otras gestiones sobre módulos}
	\begin{itemize}
		\item \texttt{(list-defmodules)}: muestra todos los módulos definidos en el sistema.
		\item \texttt{(ppdefmodule <módulo>)}: muestra la definición de un módulo dado.
	\end{itemize}
\end{frame}

\begin{frame}
\titlepage
\end{frame}
\end{document}