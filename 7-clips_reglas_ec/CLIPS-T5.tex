\documentclass[usenames,dvipsnames,aspectratio=169]{beamer}
\usepackage[utf8]{inputenc}
\usepackage[spanish]{babel}
\usepackage{csquotes}
\usepackage[skip=0pt]{caption}
\captionsetup[figure]{labelformat=empty}
\usepackage{booktabs}
\usepackage{tabularx, colortbl}
\usepackage{setspace}
\usepackage{hyperref}
\usetheme{uniud}

%%% Some useful commands
% pdf-friendly newline in links
\newcommand{\pdfnewline}{\texorpdfstring{\newline}{ }} 
% Fill the vertical space in a slide (to put text at the bottom)
\newcommand{\framefill}{\vskip0pt plus 1filll}
\renewcommand{\emph}[1]{\textbf{\textcolor{UniGold}{#1}}}


\title[]{{\Large Sistemas Inteligentes\\CLIPS}\\[0.2cm]Tema 5: Reglas II}

\date[]{Marzo de 2023}
\author[Aurora Esteban]{\texorpdfstring{
    \begin{minipage}{0.47\linewidth}
        Aurora Esteban Toscano
        \pdfnewline
        \texttt{aestebant@uco.es}
    \end{minipage}
    \hfill
    \begin{minipage}{0.47\linewidth}
        José Manuel Alcalde Llergo
        \pdfnewline
        \texttt{i72alllj@uco.es}
    \end{minipage}
    }{Aurora Esteban Toscano}
}
\institute{Grado en Ingeniería Informática, Universidad de Córdoba}

\begin{document}

\begin{frame}
\titlepage
\end{frame}

\begin{frame}{Objetivos}
	\begin{itemize}
		\item Conocer las posibilidades de configuración del antecedente a través de los \textit{Elementos Condicionales} de CLIPS.
		\item Aplicar los Elementos Condicionales tanto en hechos ordenados como en no ordenados.
	\end{itemize}
\end{frame}

\begin{frame}{Introducción a las reglas en CLIPS}
	Si el \textbf{antecedente} es cierto según los hechos almacenados en la \textbf{base de afirmaciones}, entonces \textbf{pueden} realizarse las acciones especificadas en el \textbf{consecuente}.
	\begin{block}{\small Estructura de las reglas}
		\centering\small
		\textbf{Antecedente} $\Longrightarrow$ \textbf{Consecuente}
	\end{block}
	\begin{itemize}
		\item Antecedente: cero o más \textit{cláusulas} que deben cumplirse para que la regla pueda ejecutarse (dispararse).
		\begin{itemize}
			\item Entidad patrón: elementos en los que se basa la activación de una regla. Pueden ser hechos ordenados, plantillas e instancias de clases.
			\item Elemento condicional (EC): son cada una de las condiciones que pueden encadenarse para componer el antecedente de la regla. 
		\end{itemize}
		\item Consecuente: cero o más \textit{acciones} que se llevarán a cabo si la regla se dispara.
		\begin{itemize}
			\item Entre esas acciones puede estar crear (afirmar) más hechos, eliminar hechos obsoletos, llegar a conclusiones finales...
		\end{itemize}
	\end{itemize}
\end{frame}

\begin{frame}{Tipos de Elementos Condicionales}
	Los EC se basan en aplicar restricciones sobre entidades patrón y en la concatenación de sus resultados para construir el antecedente de la regla.
	\begin{itemize}
		\item \textbf{EC patrón:} verificar los elementos de una entidad patrón.
		\item \textbf{EC test:} comprobar el valor devuelto por una función.
		\item \textbf{EC and:} comprobar si en una cadena de EC todos se cumplen.
		\item \textbf{EC or:} comprobar si en una cadena de EC al menos uno se cumple.
		\item \textbf{EC not:} invertir el resultado del EC que contiene.
		\item \textbf{EC exists:} comprobar si los EC que contiene se satisfacen por algún conjunto de hechos.
		\item \textbf{EC forall:} comprobar si los EC que contiene se satisfacen para toda ocurrencia de otro EC.
		\item \textbf{EC logical:} asegurar el mantenimiento de verdad para hechos que se basan en la presencia de otros hechos.
	\end{itemize}
\end{frame}

\begin{frame}[fragile]{EC patrón}
	Los EC patrón se usan para verificar si se cumplen campos de una entidad patrón en base a ciertos tipos de restricciones.
	\begin{itemize}
		\item Tipos de restricciones: literales, comodines, variables, sentencias conectivas, predicados, valores devueltos, direcciones de hechos.
	\end{itemize}
	\hfil
	\begin{minipage}{.48\textwidth}
		\footnotesize
		\begin{block}{Estructura: hechos ordenados}
			\begin{verbatim}
(defrule <nombre> [comentario] [propiedades]
    (<hecho ordenado> <restriccion>*)
    =>
    <consecuente>
)
			\end{verbatim}
		\end{block}
	\end{minipage}
	\hfill
	\begin{minipage}{.48\textwidth}
		\footnotesize
		\begin{block}{Estructura: hechos no ordenados}
			\begin{verbatim}
(defrule <nombre> [comentario] [propiedades]
    (<nombre plantilla>
         (<campo> <restriccion>)*
    )
    =>
    <consecuente>
)
			\end{verbatim}
		\end{block}
	\end{minipage}
\end{frame}

\begin{frame}[fragile]{EC patrón: restricciones literales}
	\begin{itemize}
		\item Verifican que la entidad patrón toma un valor exacto.
		\item Restricciones más básica, basadas únicamente en constantes.
	\end{itemize}
	\begin{minipage}{.4\linewidth}
	\begin{exampleblock}{\footnotesize Por ejemplo}
		\scriptsize
		\begin{verbatim}
(deffacts prueba1
    (ejemplo 1.0 azul "rojo")
    (ejemplo 1 azul)
    (ejemplo 1 azul rojo)
    (ejemplo 1 azul ROJO)
    (ejemplo 1 azul rojo 6.9)
    (ejemplo 1.0 amarillo ROJO 33)
    (ejemplo 3 rojo amarillo 9)
)
(defrule buscar-literal
    (ejemplo 1 azul rojo) =>
)
		\end{verbatim}
	\end{exampleblock}
	\end{minipage}
	\hfill
	\begin{minipage}{.5\linewidth}
	\begin{exampleblock}{\footnotesize Por ejemplo}
		\scriptsize
		\begin{verbatim}
(deftemplate persona
     (slot nombre)
     (slot edad)
     (multislot amigos))
(deffacts prueba2
     (persona (nombre Aurora) (edad 26))
     (persona (nombre Juan) (edad 20))
     (persona (nombre Jose) (edad 20)
              (amigos Adela Jorge))
     (persona (nombre Adela) (edad 9)
              (amigos Jose))
     (persona (nombre Carlos) (edad 40)
              (amigos María Antonio Lourdes)))
(defrule buscar-edad (persona (edad 20)) => )
		\end{verbatim}
	\end{exampleblock}
\end{minipage}
\end{frame}

\begin{frame}[fragile]{EC patrón: restricciones comodín I}
	\begin{itemize}
		\item Indican que cualquier valor en la posición indicada de la entidad patrón es válido para la regla.
		\item Dos tipos:
		\begin{itemize}
			\item Comodín monocampo \texttt{?}: Busca exactamente un campo.
			\item Comodín multicampo \texttt{\$?}: Busca cualquier número de campos (incluso 0).
		\end{itemize}
	\end{itemize}
		\begin{minipage}{.4\linewidth}
		\begin{exampleblock}{Por ejemplo}
			\footnotesize
			\begin{verbatim}
(defrule buscar-comodin
     (ejemplo ? azul rojo $?) =>
)
			\end{verbatim}
		\end{exampleblock}
	\end{minipage}
	\hfill
	\begin{minipage}{.5\linewidth}
		\begin{exampleblock}{Por ejemplo}
			\footnotesize
			\begin{verbatim}
(defrule buscar-un-amigo
     (persona (amigos ?)) =>
)
			\end{verbatim}
		\end{exampleblock}
	\end{minipage}
\end{frame}

\begin{frame}[fragile]{EC patrón: restricciones comodín II}
	Los EC patrón literal y comodín se pueden combinar para especificar restricciones complejas.
	\begin{itemize}
		\item Buscar un literal en cualquier posición: \texttt{(... \$? <literal> \$? ...)}
		\begin{exampleblock}{Por ejemplo}
			\footnotesize
			\texttt{(defrule encontrar-literal (ejemplo \$? AMARILLO \$?) => )}\\
			Emparejaría con \begin{verbatim}(ejemplo AMARILLO), (ejemplo 2.5 AMARILLO),
				(ejemplo "rojo" AMARILLO 89), (ejemplo "rojo" 33 AMARILLO)...
		\end{verbatim}
		\end{exampleblock}
		\item Buscar un multicampo con al menos un elemento: \texttt{(... ? \$?)}
		\begin{exampleblock}{Por ejemplo}
			\footnotesize
			\texttt{(defrule al-menos-un-amigo (persona (amigos ? \$?)))}
		\end{exampleblock}
	\end{itemize}
\end{frame}

\begin{frame}{EC patrón: restricciones variables I}
	\begin{minipage}{.45\textwidth}
	\begin{itemize}
		\item Almacenan un valor para luego emplearse en otros EC del antecedente o en el consecuente de una regla.
		\item Dos tipos:
		\begin{itemize}
			\item Variable monocampo \texttt{?<nombre>}: almacena exactamente un campo.
			\item Variable multicampo \texttt{\$?<nombre>}: almacena cualquier número de campos consecutivos.
		\end{itemize}
	\end{itemize}
	\end{minipage}
	\hfill
	\begin{minipage}{.45\textwidth}
		\begin{block}{OjO}
			La primera vez que la variable aparece actúa como un comodín, pero su valor queda ligado al valor del campo. Si la variable vuelve a aparecer, debe coincidir el valor ligado.
		\end{block}
		\begin{block}{OjO'}
			El ámbito de la variable sólo abarca la regla en la que aparece.
		\end{block}
	\end{minipage}
\end{frame}

\begin{frame}[fragile]{EC patrón: restricciones variables II}
	\small
	\begin{minipage}{.45\textwidth}
		\begin{exampleblock}{Por ejemplo}
			\begin{verbatim}
(defrule encontrar-terna
   (ejemplo ?x ?y ?z)
   =>
   (printout t ?x " : " ?y " : " ?z crlf)
)
			\end{verbatim}
		\end{exampleblock}
	\end{minipage}
	\hfill
	\begin{minipage}{.45\textwidth}
		\begin{exampleblock}{Por ejemplo}
			\begin{verbatim}
(defrule mis-amigos
    (persona (nombre ?n) (amigos $?a))
    =>
    (printout t "Me llamo " ?n " y mis
    amigos son " $?a crlf)
)
			\end{verbatim}
		\end{exampleblock}
	\end{minipage}
\end{frame}

\begin{frame}[fragile]{EC patrón: sentencias conectivas I}
	\begin{itemize}
		\item Permiten modificar EC patrón de tipo \textit{restricción} mediante conectores lógicos básicos: negación ($\sim$), y lógico (\texttt{\&}), o lógico (\texttt{|}).
		\begin{block}{Estructura}
			\begin{verbatim}
	(... <restricción simple> <|/&> <restricción simple> ...)
	<restricion simple> :: [~] <literal | var. monocampo | var. multicampo>
			\end{verbatim}
		\end{block}
		\item Precedencia: $\sim$ > \texttt{\&} > \texttt{|}
		\begin{itemize}
			\item Excepción: en situaciones compuestas de \texttt{<variable>\&<restriccion>*}, la variable se trata a parte.\\
			\begin{center}
				\texttt{?x\&rojo|azul} $\equiv$ \texttt{?x\&(rojo|azul)}
			\end{center}
		\end{itemize}
	\end{itemize}
\end{frame}

\begin{frame}[fragile]{EC patrón: sentencias conectivas II}
	\footnotesize
	\begin{exampleblock}{Por ejemplo}
		\begin{verbatim}
(defrule busca-rojo (ejemplo ? azul rojo|ROJO|"rojo" $?)  =>  )
(defrule excluye-azul
  (ejemplo ? ?x&~azul $?) => (printout t "No soy azul, soy " ?x crlf)
)
		\end{verbatim}
	\end{exampleblock}
	\begin{exampleblock}{Por ejemplo}
		\begin{verbatim}
(defrule no-profes
   (persona (nombre ?n&~Aurora&~Carlos))
   =>
   (printout t ?n " no es profesor" crlf)
)
		\end{verbatim}
	\end{exampleblock}
\end{frame}

\begin{frame}{EC patrón: predicados I}
	\begin{itemize}
		\item Restringen el campo según el resultado de una expresión lógica $\rightarrow$ la función debe devolver no \texttt{FALSE} para que el patrón evaluado empareje con la regla.
		\begin{block}{Estructura}
			\texttt{(... <restricción simple> : <función> ...)}
		\end{block}
		\item Uso típico: variable + conectiva \texttt{\&} + EC patrón predicado.
		\item Algunas funciones de CLIPS que devuelven expresiones lógicas:
		\begin{itemize}
			\item Tipos de datos: \texttt{integerp}, \texttt{floatp}, \texttt{numberp}, \texttt{symbolp}, \texttt{stringp}, \texttt{lexemep}
			\item Booleana: \texttt{eq}, \texttt{neq}, \texttt{and}, \texttt{or}, \texttt{not}
			\item Lógica mates: \texttt{=}, \texttt{<>}, \texttt{<}, \texttt{<=}, \texttt{>}, \texttt{>=}, \texttt{evenp}, \texttt{oddp}
		\end{itemize}
	\end{itemize}
\end{frame}

\begin{frame}[fragile]{EC patrón: predicados II}
	\footnotesize
	\begin{exampleblock}{Por ejemplo}
		\begin{verbatim}
(defrule busca-cadena (ejemplo ? ? ?x&:(stringp ?x) $?) => )
(defrule buca-1 (ejemplo ?x&:(= ?x 1) $?) => )
(defrule buca-1-mal (ejemplo ?x&:(eq ?x 1) $?) => )
		\end{verbatim}
	\end{exampleblock}
	\begin{exampleblock}{Por ejemplo}
		\begin{verbatim}
(defrule busca-jovenes
      (persona (nombre ?n) (edad ?x&:(> ?x 15)&:(< ?x 25)))
      =>
      (printout t ?n " es joven" crlf)
)
		\end{verbatim}
	\end{exampleblock}
\end{frame}

\begin{frame}[fragile]{EC patrón: valores devueltos}
	\begin{itemize}
		\item Restringe un campo en base al valor devuelto por una función $\rightarrow$ el resultado debe ser uno de los tipos de datos primitivos.
		\item El resultado de la función actuará como un literal limitando los valores de los EC a ese resultado.
		\begin{block}{Estructura}
			\texttt{(... =<función> ...)}
		\end{block}
		\item Algunas funciones numéricas de CLIPS: \texttt{+}, \texttt{-}, \texttt{*}, \texttt{/}, \texttt{div}, \texttt{mod}, \texttt{sqrt}, \texttt{**}, \texttt{round}, \texttt{abs}, \texttt{max}, \texttt{min}
	\end{itemize}
	\begin{exampleblock}{\small Por ejemplo}
		\footnotesize
		\begin{verbatim}
(defrule buscar-triple
     (ejemplo  ?x   $?   =(* 3 ?x)   $?)  =>
)
		\end{verbatim}
	\end{exampleblock}
\end{frame}

\begin{frame}[fragile]{EC patrón: direcciones de hechos}
	 Para actuar sobre hechos en el consencuente de una regla es necesario capturar su dirección y guardarla en una variable.
	\begin{itemize}
		\item En el antecedente se utiliza el operador \texttt{<-}: \texttt{?dir <- <hecho>}
		\item En el consecuente se pueden aplicar las operaciones:
		\begin{itemize}
			\item Eliminación: \texttt{(retract ?dir)}
			\item Modificación*: \texttt{(modify ?dir (<campo> <nuevo valor>))}
			\item Duplicación*: \texttt{(duplicate ?dir (<campo> <nuevo valor>))}\\
			* Solo para hechos no ordenados
		\end{itemize}
	\end{itemize}
	\begin{exampleblock}{\footnotesize Por ejemplo}
		\scriptsize
		\begin{verbatim}
(defrule comenzar
     ?h <- (iniciar_programa)
     =>
     (retract ?h) (printout t "Iniciando..." crlf)
)
		\end{verbatim}
	\end{exampleblock}
\end{frame}

\begin{frame}{Ejercicio: \texttt{pila.clp}}
	\begin{itemize}
		\item Recupera el ejercicio sobre pilas y colas trabajado la semana pasada.
		\item Añade una modificación para que sólo admita datos de tipo numérico.
		\item Añade una modificación para que la pila se rellene automáticamente dado un dato inicial y cuántos elementos más añadir.\\
	\end{itemize}
	Por ejemplo, si se parte de los hechos \texttt{(pila)}, \texttt{(dato 1)} y \texttt{(elementos 5)}, el resultado final será:\\
	\texttt{(pila 5 4 3 2 1)}
\end{frame}

\begin{frame}[fragile]{EC test I}
	El EC test comprueba el valor devuelto por una función lógica. Se satisface si el resultado es \texttt{TRUE}. Por el contrario, no se satisface si la función devuelve \texttt{FALSE}.
	\begin{itemize}
		\item Uso típico: actúa sobre variables previamente capturadas en operaciones similares a los EC patrón predicados, pero más complejas.
	\end{itemize}
	\begin{block}{Estructura}
		\begin{verbatim}
(defrule <nombre> [comentario] [propiedades]
      (<hecho> | <restricción> |...)*
      (test <función>)
      =>
      <consecuente>
)
		\end{verbatim}
	\end{block}
\end{frame}

\begin{frame}[fragile]{EC test II}
	\begin{exampleblock}{Por ejemplo}
		\begin{verbatim}
(defrule pila-correcta
     (pila ?v ?w ?x ?y ?z)
     (test (> ?v ?w ?x ?y ?z))
     =>
     (printout t "La pila está correcta" crlf)
)
		\end{verbatim}
	\end{exampleblock}
\end{frame}

\begin{frame}{EC or, and y not I}
	Estos EC sirven para encadenar otros EC mediante operaciones lógicas.
	\begin{block}{Estructura}
		\texttt{(and|or <elemento condicional> +)}\\
		\texttt{(not <elemento condicional>)}
	\end{block}
	\begin{itemize}
		\item EC or: se satisface si se satisface cualquiera de los EC que lo compone.
		\begin{itemize}
			\item La regla se activará para tantas ocurrencias como EC dentro del or se cumplan.
		\end{itemize}
		\item EC and: se satisface si todos los EC que lo componen se satisface.
		\begin{itemize}
			\item Se puede utilizar en combinación con \textit{or} en el antecedente. Utilizarlo sólo no es necesario ya que el comportamiento por defecto del antecedente es unir todas los EC mediante y lógico.
		\end{itemize}
		\item EC not: se satisface si no se satisface el EC contenido.
		\begin{itemize}
			\item Sólo puede abarcar un EC.
			\item Uso típico: comprobar que un hecho no existe en la base de afirmaciones.
		\end{itemize}
	\end{itemize}
\end{frame}

\begin{frame}[fragile]{EC or, and y not II}
	\small
	\begin{minipage}{.48\textwidth}
		\begin{exampleblock}{Por ejemplo}
			\footnotesize
			\begin{verbatim}
(defrule desayuno-sano
     (tengo zumo)
     (or (and (tengo pan) (tengo aceite))
         (and (tengo leche) (tengo cereales)))
     (not (tengo bollo))
     =>
     (printout t "Desayuno sano")
)
			\end{verbatim}
		\end{exampleblock}
	\end{minipage}
	\hfill
	\begin{minipage}{.48\textwidth}
		\begin{exampleblock}{Por ejemplo}
			\footnotesize
			\begin{verbatim}
(defrule completa-bd
     (persona (nombre ?n1)
          (amigos $? ?n2 $?))
     (not (persona (nombre ?n2)))
     =>
     (assert (persona (nombre ?n2)
                  (amigos ?n1)))
)
			\end{verbatim}
		\end{exampleblock}
	\end{minipage}
	\textbf{Ejercicio 1:} ¿cómo podría mejorarse la regla \texttt{mis-amigos} definida anteriormente para que no muestre personas sin amigos?\\
	\textbf{Ejercicio 2:} ¿cómo se crearía una regla que inicialice la pila/cola de ejercicios anteriores en caso de que no exista?
\end{frame}

\begin{frame}{EC exists I}
	El EC exists permite comprobar si una serie de EC se satisface por \textbf{algún} conjunto de hechos.
	\begin{itemize}
		\item El comportamiento normal de CLIPS es que la regla se active para \textbf{todos} los conjuntos de hechos que la satisfagan.
	\end{itemize}
	\begin{block}{Estructura}
		\texttt{(exists <elemento condicional> +)}\\
	\end{block}
\end{frame}

\begin{frame}[fragile]{EC exists II}
	Observa la regla del fichero disponible en Moodle \texttt{superheroes.clp}. Cárgalo en Moodle y observa el comportamiento:
	\begin{minipage}{\textwidth}
		\footnotesize
		\begin{verbatim}
CLIPS> (clear)
CLIPS> (load superheroes.clp)
CLIPS> (reset)
CLIPS> (facts)
f-1     (superheroe (nombre spiderman) (estado libre))
f-2     (superheroe (nombre viuda-negra) (estado ocupado))
f-3     (superheroe (nombre drextrange) (estado libre))
f-4     (superheroe (nombre capitana-marvel) (estado ocupado))
CLIPS> (agenda)
0      salvar-dia: f-3
0      salvar-dia: f-1
		\end{verbatim}
	\end{minipage}
	\begin{exampleblock}{\footnotesize Solución}
		\scriptsize
		\begin{verbatim}
(defrule salvar-dia "Necesitamos que un súper heroe salve el día"
     (exists (superheroe (estado libre))) => (printout t "El día está salvado" crlf)
)
		\end{verbatim}
	\end{exampleblock}
\end{frame}

\begin{frame}{EC forall I}
	El EC forall permite comprobar si para toda ocurrencia del primer EC se satisface el conjunto de los siguientes EC contenidos en el \texttt{forall}.
	\begin{block}{Estructura}
		\texttt{(forall <elemento condicional> <elemento condicional>+)}
	\end{block}
\end{frame}

\begin{frame}[fragile]{EC forall II}
	Descarga y observa la estructura del fichero \texttt{estudiantes.clp}.
	¿Qué es lo que falta para que se active la regla \texttt{todos-limpios}?
	\begin{exampleblock}{La regla}
		\begin{verbatim}
(defrule todos-limpios
     (forall (estudiante ?nombre)
             (aprobado lengua ?nombre) (aprobado matematicas ?nombre)
             (aprobado historia ?nombre)
     )
     =>
     (printout t "Todos mis estudiantes pasan de curso" crlf)
)
		\end{verbatim}
	\end{exampleblock}
\end{frame} 

\begin{frame}{EC logical I}
	El EC logical asegura el \textit{mantenimiento de verdad} de los hechos creados a partir de él.
	\begin{enumerate}
		\item Los hechos del antecedente proporcionan \textbf{soporte lógico} a los hechos que se han creado en el consecuente de una regla basada en el EC forall. $\rightarrow$ Los hechos permanecen en la base de hechos mientras permanezcan los que los soportan lógicamente.
		\item  Un hecho puede recibir soporte lógico de varios conjuntos distintos de hechos $\rightarrow$ para permanecer necesita que exista al menos uno de los conjuntos.
		\item Los EC incluidos en un EC logical están unidos implícitamente por Y lógico.
		\item Puede combinarse con EC \textit{and}, \textit{or} y \textit{not}.
		\item Sólo los primeros EC del antecedente pueden ser de tipo logical.
	\end{enumerate}
\end{frame}

\begin{frame}[fragile]{EC logical II}
	\scriptsize
	\begin{verbatim}
CLIPS> (defrule puedo-pasar (logical (or
           (and (semaforo verde) (not (viene ambulancia))) (and (semaforo rojo) (sin coches))))
    => (assert (puedo cruzar-paso)))
CLIPS> (assert (semaforo verde))
CLIPS> (run)
CLIPS> (facts)
f-1     (semaforo verde)
f-2     (puedo cruzar-paso)
CLIPS> (assert (viene ambulancia))
CLIPS> (facts)
f-1     (semaforo verde)
f-3     (viene ambulancia)
CLIPS> (retract 1 3)
CLIPS> (assert (semaforo rojo) (sin coches))
CLIPS > (run)
CLIPS> (facts)
f-4     (semaforo rojo)
f-5     (sin coches)
f-6     (puedo cruzar-paso)
CLIPS> (retract 5)
CLIPS> (facts)
f-4     (semaforo rojo)
	\end{verbatim}
\end{frame}
 
\begin{frame}
\titlepage
\end{frame}
\end{document}