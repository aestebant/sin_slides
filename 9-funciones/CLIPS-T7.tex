\documentclass[usenames,dvipsnames,aspectratio=169]{beamer}
\usepackage[utf8]{inputenc}
\usepackage[spanish]{babel}
\usepackage{csquotes}
\usepackage[skip=0pt]{caption}
\captionsetup[figure]{labelformat=empty}
\usepackage{booktabs}
\usepackage{tabularx, colortbl}
\usepackage{setspace}
\usepackage{hyperref}
\usetheme{uniud}

%%% Some useful commands
% pdf-friendly newline in links
\newcommand{\pdfnewline}{\texorpdfstring{\newline}{ }} 
% Fill the vertical space in a slide (to put text at the bottom)
\newcommand{\framefill}{\vskip0pt plus 1filll}
\renewcommand{\emph}[1]{\textbf{\textcolor{UniGold}{#1}}}


\title[]{{\Large Sistemas Inteligentes\\CLIPS}\\[0.2cm]Tema 7: Funciones y funciones genéricas}

\date[]{Abril de 2023}
\author[Aurora Esteban]{\texorpdfstring{
    \begin{minipage}{0.47\linewidth}
        Aurora Esteban Toscano
        \pdfnewline
        \texttt{aestebant@uco.es}
    \end{minipage}
    \hfill
    \begin{minipage}{0.47\linewidth}
        José Manuel Alcalde Llergo
        \pdfnewline
        \texttt{i72alllj@uco.es}
    \end{minipage}
}{Aurora Esteban Toscano}
}
\institute{Grado en Ingeniería Informática, Universidad de Córdoba}

\begin{document}

\begin{frame}
\titlepage
\end{frame}

\begin{frame}{Objetivos}
	\begin{itemize}
		\item Aprender a implementar funciones en CLIPS.
		\item Aprender a implementar funciones genéricas en CLIPS.
		\item Conocer las principales funciones y acciones del sistema en CLIPS.
	\end{itemize}
\end{frame}

\begin{frame}[fragile]{Introducción a las funciones en CLIPS}
	Las funciones en CLIPS agrupan una secuencia de acciones, o expresiones, a ejecutar de forma secuencial mediante una única llamada a función.
	
	Típicamente reciben una entrada en forma de parámetros de entrada y producen una salida en base a dichos parámetros.
	
	\begin{block}{Definición}
		\begin{verbatim}
(deffunction <nombre> [comentario]
      (<parámetro regular>* [parámetro comodín])
      <acción>*
)
<parámetro regular>:: <variable mono-campo>
<parámetro irregular>:: <variable multi-campo>
		\end{verbatim}
	\end{block}
\end{frame}

\begin{frame}{Propiedades de las funciones}
	\begin{itemize}
		\item Las funciones reciben una lista de cero o más parámetros regulares: \textbf{obligatorios}.
		\item Las funciones pueden recibir parámetros adicionales recogidos en el parámetro comodín: \textbf{opcionales}.
		\item Las funciones deben tener un nombre único y declararse antes de ser llamadas.
		\item Las funciones devuelven el resultado de evaluar su última acción y \texttt{FALSE} si no tienen ninguna acción.
		\item Las funciones pueden llamarse a sí mismas: útil para implementar un \textbf{comportamiento recursivo}.
	\end{itemize}
\end{frame}

\begin{frame}[fragile]{Ejemplo}
	\begin{verbatim}
CLIPS> (deffunction formatea-args
    (?a ?b $?c)
    (printout t "Tengo dos parametros obligatorios: " ?a " y " ?b
    ". Ademas de " (length$ $?c) " opcionales: " $?c crlf)
)
CLIPS> (formatea-args hola adios)
Tengo dos parametros obligatorios: hola y adios. Ademas de 0 opcionales: ()
CLIPS> (formatea-args hola adios "buenas tardes" 33)
Tengo dos parametros obligatorios: hola y adios. Ademas de 2 opcionales: ("buenas
tardes" 33)
CLIPS>
	\end{verbatim}
\end{frame}

\begin{frame}[fragile]{Ejemplo recursivo}
	\small
	\begin{verbatim}
CLIPS> (deffunction factorial
    (?a)
    (if (or (not (integerp ?a)) (< ?a 0)) then
       (printout t "Factorial error: " ?a crlf)
     else
       (if (= ?a 0) then 1
        else (* ?a (factorial (- ?a 1)))
       )
   )
)
CLIPS> (factorial 4)
24
CLIPS> (factorial -3)
Factorial error: -3
CLIPS> (factorial adios)
Factorial error: adios
	\end{verbatim}
\end{frame}

\begin{frame}{Otras gestiones sobre funciones}
	\begin{itemize}
		\item \texttt{(list-deffunctions)}: muestra todas las funciones definidas en el sistema.
		\item \texttt{(ppdeffunction <función>)}: muestra la definición de una función dada.
		\item \texttt{(undeffunction <función>)}: borra la función dada.
	\end{itemize}
\end{frame}

\begin{frame}{Funciones genéricas}
	Las funciones genéricas es la forma que tiene CLIPS de programar funciones que variarán su comportamiento dependiendo de los parámetros con los que se llamen: \textbf{sobrecarga de funciones}.
	\begin{itemize}
		\item Las funciones genéricas se componen de una \textbf{cabecera} y de una lista de \textbf{métodos}: cada método tiene una lista de parámetros diferentes.
		\item Declaración de la cabecera:
		\begin{itemize}
			\item Explícitamente: usando el constructor \texttt{defgeneric}.
			\item Implícitamente: definiendo al menos un método de la función mediante \texttt{defmethod}.
		\end{itemize}
		\item Los métodos tienen un comportamiento similar a las funciones "normales" de CLIPS, con la salvedad de estar agrupados en una misma \textit{familia} y las ventajas:
		\begin{itemize}
			\item Pueden sobrescribir o ampliar funciones del sistema.
			\item Pueden tener restricciones sobre los parámetros que reciben.
		\end{itemize}
	\end{itemize}
\end{frame}

\begin{frame}[fragile]{Funciones genéricas: definición}
	\footnotesize
	\begin{block}{Cabecera}
		\texttt{(defgeneric <nombre> [comentario])}
	\end{block}
	\begin{block}{Métodos}
		\begin{verbatim}
(defmethod <nombre> [índice] [comentario]
     (<restriccion mono-campo>* [restriccion multi-campo])
     <acción>*
)
<restriccion mono-campo> :: <variable mono-campo> | (<variable
      mono-campo> <tipo>* [consulta])
<restriccion multi-campo> :: <variable multi-campo> | (<variable
      multi-campo> <tipo>* [consulta])
<tipo> :: INTEGER | FLOAT | NUMBER | SYMBOL | STRING | LEXEME
<consulta> :: <variable global> | <llamada a función>
		\end{verbatim}
	\end{block}
\end{frame}

\begin{frame}{Propiedades de las funciones genéricas}
	\begin{itemize}
		\item Identificación de un método:
		\begin{itemize}
			\item Nombre + lista de parámetros.
			\item Nombre + índice.
		\end{itemize}
		\item A cada método se le asigna un índice único dentro de una función genérica.
		\item Si se define un método con el mismo nombre y parámetros que otro existente, se sobrescribe el más antiguo.
		\item Reemplazar un método por otro con una definción de parámetros distinta: especificar el índice del método a reemplazar al definir el nuevo.
		\item Restricciones sobre parámetros:
		\begin{itemize}
			\item De \textit{tipo}: limita los tipos de datos admisibles.
			\item De \textit{consulta}: comprobación lógica sobre el argumento que debe devolver distinto de \texttt{FALSE}.
			\item Las restricciones se evalúan de izquierda a derecha.
			\item Si una restricción no se cumple, dejan de comprobarse las que siguen.
		\end{itemize}
	\end{itemize}
\end{frame}

\begin{frame}{Ejemplos de restricciones}
	\begin{itemize}
		\item Ningún parámetro: \texttt{(defmethod m1 ())}
		\item Un parámetro de cualquier tipo: \texttt{(defmethod m1 (?a))}
		\item Un parámetro entero: \texttt{(defmethod m1 ((?a INTEGER)))}
		\item Un parámetro entero o símbolo: \texttt{(defmethod m1 ((?a INTEGER SYMBOL)))}
		\item Un parámetro entero y par: \texttt{(defmethod m1 ((?a INTEGER (evenp ?a))))}
		\item Un parámetro entero o símbolo y otro de cualquier tipo: \texttt{(defmethod m1 ((?a INTEGER SYMBOL) ?b))}
	\end{itemize}
\end{frame}

\begin{frame}[fragile]{Ejemplo}
	\small
	\begin{verbatim}
CLIPS> (defmethod + ((?a STRING) (?b STRING))
    (str-cat ?a ?b))
CLIPS> (defmethod + ((?a STRING) (?b NUMBER))
    (printout t "No se que hacer con eso." crlf))
CLIPS> (list-defmethods)
+ #SYS1  (NUMBER) (NUMBER) ($? NUMBER)
+ #2  (STRING) (STRING)
+ #3  (STRING) (NUMBER)
CLIPS> (+ "Hola " "Mundo")
"Hola Mundo"
	\end{verbatim}
	\begin{itemize}
		\item \textbf{Ejercicio:} ¿Qué pasará si se llama a la función con (+ <número> <cadena>)?
		\item \textbf{Ejercicio:} ¿Cómo se extendería la función para que si se llama con dos símbolos los devuelva concatenados con un espacio en medio?
	\end{itemize}
\end{frame}

\begin{frame}{\textit{Generic dispatch}}
	\textit{Generic dispatch} o \textit{ejecución genérica}: proceso por el que CLIPS decide qué método de la función genérica ejecutar según la llamada realizada.
	\begin{itemize}
		\item Un método es aplicable a una llamada de función genérica si:
		\begin{enumerate}
			\item Su nombre coincide con el de la función genérica.
			\item Acepta al menos tantos argumentos como se pasan en la llamada.
			\item Cada argumento satisface las restricciones de parámetro correspondientes.
		\end{enumerate}
		\item La precedencia entre dos métodos se determina mediante sus restricciones de parámetro: el método con las restricciones más específicas tiene más precedencia.
	\end{itemize}
\end{frame}

\begin{frame}[fragile]{Ejemplo}
	\small
	\begin{verbatim}
CLIPS> (defmethod m1 () m1-1)
CLIPS> (defmethod m1 (?a) m1-2)
CLIPS> (defmethod m1 ((?a INTEGER)) m1-3)
CLIPS> (defmethod m1 ((?a INTEGER SYMBOL)) m1-4)
CLIPS> (defmethod m1 ((?a INTEGER STRING)) m1-5)
CLIPS> (defmethod m1 ((?a INTEGER (evenp ?a))) m1-6)
CLIPS> (m1)
m1-1
CLIPS> (m1 2.5)
m1-2
CLIPS> (m1 25)
m1-3
CLIPS> (m1 hola)
m1-4
CLIPS> (m1 22)
m1-6
	\end{verbatim}
\end{frame}

\begin{frame}{Otras gestiones sobre funciones genéricas}
	\begin{itemize}
		\item \texttt{(list-defgenerics)}: muestra todas las funciones genéricas definidas en el sistema.
		\item \texttt{(ppdefgeneric <función>)}: muestra la definición de una función genérica dada.
		\item \texttt{(undefgeneric <función>)}: borra la función genérica dada, incluyendo todos sus métodos.
		\item \texttt{(list-defmethods <función>)}: muestra los métodos definidos para una función genérica.
		\item \texttt{(ppdefmethod <función> <índice>)}: muestra la definición de un método de una función genérica.
		\item \texttt{(undefmethod <función> <índice>)}: borra un método de una función genérica dada.
	\end{itemize}
\end{frame}

\begin{frame}{Funciones del sistema}
	CLIPS predefine múltiples funciones que pueden usarse para las acciones que se programen tanto en funciones del usuario como en funciones genéricas.\\[.5cm]
	
	\begin{minipage}{.35\linewidth}
		\begin{itemize}
			\item Funciones predicado
			\item Funciones matemáticas
			\item Funciones multi-campo
			\item Funciones de cadena
			\item Funciones de entrada/salida
			\item Funciones procedurales
		\end{itemize}
	\end{minipage}
	\hfill
	\begin{minipage}{.55\linewidth}
		\begin{block}{OjO}
			Recuerda que tanto las funciones como los métodos de las funciones genéricas pueden contener varias acciones en su cuerpo. Devolverán el resultado de su última acción o \texttt{FALSE} si no hay acciones.
		\end{block}
	\end{minipage}
\end{frame}

\begin{frame}{Funciones del sistema: predicado}
	\begin{itemize}
		\item Tipo de dato: \texttt{(integerp | floatp | numberp | symbolp | stringp | lexemep <expresión>)}
		\item Es multi-campo: \texttt{(multifieldp <expresión>)}
		\item Son misma expresión o distinta: \texttt{(eq | neq <expresión> <expresión>+)}
		\item Número par o impar: \texttt{(evenp | oddp <expresión>)}
		\item Comparaciones numéricas: \texttt{(= | <> | < | <= | > | >= <expresión> <expresión>+)}
		\item Lógica booleana: \texttt{(not <expresión>)}, \texttt{(and | or <expresión>+)}
	\end{itemize}
\end{frame}

\begin{frame}{Funciones del sistema: op. matemáticas}
	\begin{itemize}
		\item Suma, resta, multiplicación o división: \texttt{(+ | - | * | / <expresión> <expresión>+)}
		\item División entera: \texttt{(div <expresión> <expresión>)}
		\item Resto o módulo de la división: \texttt{(mod <expresión> <expresión>)}
		\item Raíz cuadrada: \texttt{(sqrt <expresión>)}
		\item Potencia: \texttt{(** <expresión> <expresión>)}
		\item Redondeo: \texttt{(round <expresión>)}
		\item Valor absoluto: \texttt{(abs <expresión>)}
		\item Máximo o mínimo de un conjunto: \texttt{(max | min <expresión>+)}
	\end{itemize}
\end{frame}

\begin{frame}{Funciones del sistema: multicampos I}
	\begin{itemize}
		\item Crear un multi-campo: \texttt{(create\$ <expresión>*)}
		\item Longitud de un multi-campo: \texttt{(length\$ <expresión multi>)}
		\item Obtener $n$-ésimo: \texttt{(nth\$ <posición> <expresión multi>)}
		\item Es miembro: \texttt{(member\$ <expresión> <expresión multi>)}
		\item Es subconjunto: \texttt{(subsetp <expresión multi> <expresión multi>)}
		\item Extraer subconjunto: \texttt{(subseq\$ <expresión multi> <posición ini> <posición fin>)}
		\item Extraer primero: \texttt{(first\$ <expresión multi>)}
		\item Todos menos el primero: \texttt{(rest\$ <expresión multi>)}
	\end{itemize}
\end{frame}

\begin{frame}{Funciones del sistema: multicampos II}
	\begin{itemize}
		\item Transformar una cadena en multicampo: \texttt{(explode\$ <cadena>)}
		\item Tranformar un multicampo en cadena: \texttt{(implode\$ <expresión multi>)}
		\item Insertar: \texttt{(insert\$ <expresión multi> <posición ini> <expresión>+)}
		\item Borrar: \texttt{(delete\$ <expresión multi> <posición ini> <posición fin>)}
		\item Reemplazar: \texttt{(replace\$ <expresión multi> <posición ini> <posición fin> <expresión>+)}
	\end{itemize}
\end{frame}

\begin{frame}{Funciones del sistema: cadenas}
	\begin{itemize}
		\item Concatenar en una cadena: \texttt{(str-cat <expresión>+)}
		\item Concatenar en un símbolo: \texttt{(sym-cat <expresión>+)}
		\item Longitud de una cadena o símbolo: \texttt{(str-length <expresión>)}
		\item Extraer cadena: \texttt{(sub-string <posición ini> <posición fin> <cadena>)}
		\item Comparación lógica de cadenas o símbolos: \texttt{(str-compare <expresión> <expresión>)}
	\end{itemize}
\end{frame}

\begin{frame}{Funciones del sistema: entrada/salida}
	\begin{itemize}
		\item Abrir un fichero: \texttt{(open <fichero> <nombre> <modo>)}
		\begin{itemize}
			\item Posibles modos: "\texttt{r}" (lectura), "\texttt{w}" (escritura), "\texttt{r+}" (lectura y escritura), "\texttt{a}" (añadir al final),
		\end{itemize}
		\item Cerrar un fichero: \texttt{(close <nombre>)}
		\item Escribir en un fichero: \texttt{(printout <nombre> <expresión>+)}
		\begin{itemize}
			\item Escritura por consola: \texttt{(printout t <expresión>+)}
		\end{itemize}
		\item Leer de un fichero: \texttt{(read <nombre>)}
		\begin{itemize}
			\item Leer desde la consola la entrada del usuario \texttt{(read)}
		\end{itemize}
		\item Leer una línea de un fichero: \texttt{(readline <nombre>)}
	\end{itemize}
\end{frame}

\begin{frame}{Funciones del sistema: procedurales}
	\begin{itemize}
		\item Definir una variable: \texttt{(bind <variable> <expresión>)}
		\item Sentencia condicional: \texttt{(if <expresión> then <acción>* [else <acción>*])}
		\item Sentencia condicional múltiple: \texttt{(switch <expresión prueba> (case <expresión comprobación then <acción>*)+ [(default <acción>*)])}
		\item Bucle mientras: \texttt{(while <expresión> [do] <acción>*)}
		\item Bucle para: \texttt{(loop-for-count <rango> [do] <acción>*)}
		
		\texttt{<rango>:: <idx fin> | (<variable> <idx fin>) | (<variable> <idx ini> <idx fin>)}
		\item Salir de una función: \texttt{(return [<expresión>])}
		\item Salir de un bucle: \texttt{(break)}
	\end{itemize}
\end{frame}

\begin{frame}{Ejemplos}
	\begin{itemize}
		\item Estudia un ejemplo de uso de bucles en funciones en el fichero \texttt{bucles.clp} disponible en Moodle.
		\item Estudia un ejemplo de petición de datos por pantalla en el fichero \texttt{transformaciontemporal.clp} disponible en Moodle.
	\end{itemize}
\end{frame}

\begin{frame}{Ejercicios}
	\begin{enumerate}
		\item Crear una función genérica para extraer N elementos de una cadena o de un valor multicampo. La función recibirá como parámetros en los dos casos el índice del primer y último elemento a extraer y como tercer parámetro la cadena o multicampo sobre los que realizar la extracción.
		
		\item Crear una función que, utilizando recursividad, calcule el $n$-ésimo número par (considerando el 2 como primer par). La función recibe un parámetro, la posición que se desea obtener.
		
		\item Crear una función que, utilizando recursividad, calcule el $n$-ésimo número impar (considerando el 1 como primer impar).
		
		\item Crear una función que, utilizando recursividad, calcule la suma de los $n$ primeros números pares.
		
		\item Crear una función que, utilizando recursividad, calcule el $n$-ésimo término de la sucesión de números triangulares.
	\end{enumerate}
\end{frame}

\begin{frame}
\titlepage
\end{frame}
\end{document}