\documentclass[usenames,dvipsnames,aspectratio=169]{beamer}
\usepackage[utf8]{inputenc}
\usepackage[spanish]{babel}
\usepackage{csquotes}
\usepackage[skip=0pt]{caption}
\captionsetup[figure]{labelformat=empty}
\usepackage{booktabs}
\usepackage{tabularx, colortbl}
\usepackage{setspace}
\usepackage{hyperref}
\usetheme{uniud}

%%% Some useful commands
% pdf-friendly newline in links
\newcommand{\pdfnewline}{\texorpdfstring{\newline}{ }} 
% Fill the vertical space in a slide (to put text at the bottom)
\newcommand{\framefill}{\vskip0pt plus 1filll}
\renewcommand{\emph}[1]{\textbf{\textcolor{UniGold}{#1}}}


\title[N-Reinas]{{\LARGE Sistemas Inteligentes\\CLIPS}\\[0.5cm]Tema 2: Hechos I}

\date[Marzo, 2022]{07 de marzo de 2022}
\author[A. Esteban]{\texorpdfstring{
    \begin{minipage}{0.47\linewidth}
        Aurora Esteban Toscano
        \pdfnewline
        \texttt{aestebant@uco.es}
    \end{minipage}
}{Aurora Esteban Toscano}
}
\institute{Grado en Ingeniería Informática, Universidad de Córdoba}
\begin{document}

%\setstretch{1.5}

\begin{frame}
\titlepage
\end{frame}

\begin{frame}{Objetivos}
	\begin{itemize}
		\item Introducir los \textit{hechos} en CLIPS.
		\item Entender la diferencia entre hechos ordenados y no ordenados.
		\item Operaciones sobre hechos.
	\end{itemize}
\end{frame}

\begin{frame}{Introducción a los hechos en CLIPS}
	Los hechos representan la información actual del Sistema Experto: determinan cómo actúa y se actualizan según las condiciones cambian (las distintas reglas se disparan).
	
	Elementos de un hecho:
	\begin{itemize}
		\item Índice
		\item Dirección
		\item Valor(es)
	\end{itemize}

	Tipos de hechos:
	\begin{itemize}
		\item Ordenados: son una relación de valores. Por ejemplo:\\
		\texttt{(lista huevos leche ``pimiento rojo'')}
		\item No ordenados: su estructura se determina por una plantilla. Por ejemplo:\\
		\texttt{(coche (marca Seat) (modelo Ibiza) (color amarillo))}
	\end{itemize}
\end{frame}

\begin{frame}{Afirmación de hechos}
	Dos modalidades:
	\begin{itemize}
		\item \texttt{assert}: introducir en la base de hechos los hechos dinámicamente. Por ejemplo:\\
		\texttt{(assert (a b c))}\\
		\texttt{(assert (a) (b) (c))}\\
		\texttt{(assert (a b) (c d) (e))}
		\item \texttt{deffact}: inicializar la base de hechos al principio. Por ejemplo:\\
		\texttt{(deffacts h1 ``hechos iniciales'' (a b) (c d) (e))}
		\begin{itemize}
			\item La base de hechos no se inicializará hasta ejecutar \texttt{(reset)}
		\end{itemize}
		\item OjO: no se puede insertar un hecho que ya existe.
	\end{itemize}
\end{frame}


\begin{frame}[fragile]{Afirmación de hechos no ordenados}
	La estructura de los hechos no ordenados están determinados por una plantilla que permite:
	\begin{itemize}
		\item No tener que rellenar todos los campos para todos los hechos.
		\item Rellenar los campos de forma desordenada.
		\item Especificar el tipo de valor esperado para cada campo.
		\item ... Se verán en detalle la semana que viene. 
	\end{itemize}
	\texttt{deftemplate}: definir la plantilla para los hechos de un cierto tipo. Por ejemplo:
\begin{verbatim}
(deftemplate coche ``Es un coche''
    (slot marca)
    (slot modelo)
    (slot color)
)
(assert (coche (marca Seat)))
\end{verbatim}
\end{frame}

\begin{frame}{Visualización de hechos}
	\texttt{(facts)}: muestra el estado actual de la base de hechos con formato \texttt{f-<idx> (hecho)}. También se puede utilizar para filtrar:
	\begin{itemize}
		\item \texttt{(facts <indice minimo>)}
		\item \texttt{(facts <indice minimo> <indice maximo>)}
	\end{itemize}
	\texttt{(watch facts)}: es una orden que sirve para depuración: muestra cómo los hechos son insertados o eliminados de la base de hechos:\\
	\texttt{==>}: Hecho entrando a la base de hechos.\\
	\texttt{<==}: Hecho saliendo de la base de hechos.
	\begin{itemize}
		\item Para desactivar: \texttt{(unwatch facts)}
	\end{itemize}
\end{frame}

\begin{frame}[fragile]{Modificación de hechos}
	Para modificar un hecho primero se debe capturar su dirección: operador \texttt{<-}.
	\begin{minipage}{.4\linewidth}
		\begin{exampleblock}{Por ejemplo}
		\begin{verbatim}
(defrule comenzar
   ?h <- (iniciar_programa)
    =>
   (retract ?h)
   (printout t ``Iniciando...'' crlf)
)
		\end{verbatim}
		\end{exampleblock}
	\end{minipage}
	\begin{minipage}{.5\linewidth}
		Operaciones de modificación:
		\begin{itemize}
			\item \texttt{retract}: eliminar el hecho.\\
			\texttt{(retract 1) (retract 4 7) (retract *)}
			\item Sólo para hechos ordenados:
			\begin{itemize}
				\item \texttt{modify}: modificar una o varias casillas del hecho.\\
				\texttt{(modify 1 (color rojo))}
				\item \texttt{duplicate}: duplicar el hecho cambiando al menos una casilla.
				\texttt{(duplicate 1 (color rojo))}
			\end{itemize}
		\end{itemize}
	\end{minipage}
\end{frame}

\begin{frame}{Restablecimiento del entorno CLIPS}
	Existen dos grados de restablecimiento de CLIPS:
	\begin{itemize}
		\item \texttt{(reset)}: restablece la base de hechos, eliminando los hechos creados con \texttt{assert} y volviendo a añadir los definidos en \texttt{deffacts}.
		
		\item \texttt{(clear)}: elimina la base de hechos, la base de conocimiento y los constructores definidos. Restablece completamente el sistema.
	\end{itemize}
\end{frame}

\begin{frame}{Legibilidad de hechos}
	Buenas prácticas para aumentar la legibilidad de hechos en CLIPS:
	\begin{itemize}
		\item Utilizar comentarios en los constructores \texttt{deffacts}.
		\item Utilizar espacios, saltos de línea y tabulación, no afectan al programa.
		\begin{itemize}
			\item Dentro de una cadena los espacios sí cuentan.
		\end{itemize}
		\item En los hechos ordenados, destinar el primer campo a nombrar la relación.
		\begin{itemize}
			\item Peor: (leche pan huevos), (perro), (gato)
			\item Mejor: (lista-compra leche pan huevos) (es-animal perro) (es-animal gato)
		\end{itemize}
	\end{itemize}
\end{frame}

\begin{frame}[fragile]{Ejemplo}
\begin{verbatim}
CLIPS> (clear)
CLIPS> (assert (color rojo))
<Fact-0>
CLIPS> (assert (color azul) (valor (+ 3 4)))
<Fact-2>
CLIPS> (assert (color rojo))
FALSE
CLIPS> (deftemplate estado (slot temperatura) (slot presion))
CLIPS> (assert (estado (temperatura alta) (presion baja)))
<Fact-3>
CLIPS> (facts)
f-0     (color rojo)
f-1     (color azul)
f-2 (valor 7)
f-3 (estado (temperatura alta) (presion baja))
\end{verbatim}
\end{frame}

\begin{frame}{Ejemplo: ahora tú}
	\begin{itemize}
		\item Muestra los hechos con índice >=1
		\item Muestra los hechos 1 a 2
		\item Crea un nuevo hecho que sea como el 3 pero con (temperatura baja)
		\item Elimina el hecho 1
		\item Añade un hecho (color verde)
		\item Elimina todos los hechos
	\end{itemize}
\end{frame}

\begin{frame}
\titlepage
\end{frame}
\end{document}